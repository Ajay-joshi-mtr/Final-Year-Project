\chapter{Introduction}
My project is a Native Android E-Commerce Application built in Android Studio using Kotlin/Java programming languages. The general concept of my application is to create a e-commerce app that mirrors Amazon or Alibaba.

A specific goal of this project was to learn how native development differs from hybrid development(Xamarin, Ionic, React Native etc). Hybrid development was a topic that had multiple course projects, but i never got to experience building a project based on native development in Android or iOS.
\newline 

A focus i made on my application is integrating different technologies such as Firebase, Picasso and Retrofit. Using these technologies made certain aspects of my application easier to implement. For example Picasso which is a image downloading and caching library can manage many aspects of image processing in an android environment such as handling ImageView recycling, complex image transformations with minimal memory use and automatic memory and disk caching which all have a heavy amount of coding involved to be implemented manually. \newline

Firebase is another technology i use which has a great number of features and products such as Cloud Firestore which stores and syncs data between users and devices - at global scale - using a cloud-hosted, NoSQL database.

Authentication to manage users in a secure way by offering authentication through email, password and third-party providers like GitHub, Google, Facebook and Twitter.

Realtime Database which is an efficient, low-latency solution for mobile apps that require synced states across clients in realtime.
\newline 



\newpage
\subsection{Objectives for project}

\textbf {Use a new programming language and framework} 

One of the objectives i had set for this project is to use a new language and framework, i made the choice of android and kotlin as it met all my objectives for my project. The main goal of my project was to build a native mobile application and compare it to hybrid applications that i have built throughout my course.\newline

\begin{itemize}
    \item Learn a new programming language and framework
    \item Create a Native Android Application
    \item Add Social Integration (Employee Features, Chat/Calendar etc.) 
    \item Add Statistics for Purchasing and Consumer Habits
\end{itemize}

\textbf {Create a Native Mobile Application} 
\newline

\textbf {Use new and useful technologies}
ss \newline
\textbf {Include social Media Integration}
ss \newline

\textbf {Create a Native Mobile Application} \newline

\newpage

\subsection{Sections}

\textbf {Methodology}
After setting objectives for my project, i set out to make a plan for development using Kanban boards. I used an agile approach for my development setting specific goals for each week. Validation and Testing was done using Junit.
\newline


\textbf {Technology Review}
In this section i explain the Technologies used in my project, technologies such as Firebase, Kotlin, Android Studio are explained in great detail. I will also review programming in android with kotlin compared to my experience through out my four years in college. Comparisons such as Native development vs Hybrid development, use cases for different programming languages, language features such as more use of lambda expressions and more Kotlin features that make it a great alternative to java for android development.
\newline


\textbf {System Design}
In this section i provided an explanation of the overall system architecture using UML class, sequence and interaction diagrams as well as screenshots of UI components such as how each view is formed with ImageView, TextBox etc. \newline

\textbf {System Evaluation}
  In this section i evaluate my project against the objectives i set out. \newline

\textbf {Conclusion}

 In my conclusion i evaluate my overall goals for my project and list outcomes of the project. I go over discoveries made, what I've learned and what i can improve on in the future \newline

\textbf {Github Repository} \url{https://github.com/DarrenRegan/Final-Year-Project}
\newline

    My GitHub Repository contains my Dissertation, APK file which are both available for download with a click of a button at the top of the README. 
\begin{itemize}
     \item The README contains a quick explanation of the project as well as an installation guide, Devices used in testing as well as resources used along with links to research material.
         
     \item The code for my project is located at Final-Year-Project/app/src/main/
     Java folder contains the code for activities and models
     
    \item Res folder contains the code for UI elements of the project, \newline              main/res/layout contains XML for all activities
\end{itemize}
