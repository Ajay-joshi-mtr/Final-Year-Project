\chapter{Conclusion}
This chapter serves as a conclusion to the project. In it, the objectives of the project are analysed again briefly to evaluate whether or not they were met. It also analyses any improvements and/or changes that could be added/made to the project if it was to be developed again.

\section{Objectives Review}
The main Goal and Objective of my project was to Learn Android Development, Kotlin and Android Studio. I set my requirements to implement the current recommend development methods created by Google. Everything else was an objective or extra feature if i had time. I feel confident that i achieved my goals for this Project. I spent well over 200 Hours inside Android Studio while learning how to implement and get accustomed to developing with Kotlin. Kotlin removes are large amount of boilerplate code, which for a large amount of the project slowed me down because i am so accustomed to Java development. A java method that is 20 to 30 lines of code can be reduced to a simple Lambda expression in Kotlin which infers all the logic that java would have to specify. \newline

Overall i am happy with what I've achieved with my Project, while the COVID-19 pandemic did affect how much i implemented i achieved the most important Goals of my project which were to Learn Android Development, Kotlin and Android Studio.
\newpage
\subsection{What i learned}
The main objectives for this project was to learn a new programming language and to learn and create a native android application. These two objectives are the most important as the main goal was to get insight on Android Development as Android Development was something i was debating on choosing as a main career option.
Through completing this Project i feel like i gained exactly what i wanted out of this project and i am now highly considering Android Development as my career going forward. 
\bigskip

\section{Improvements}
There are many improvements i would want to make to my application, integrating the many products from Firebase shown in the tech review, creating a product analysis, creating consumer statistics for buying habits, integrating Google Analytics for advertisements etc. All of these we're considered and even attempted but were ultimately out of scope of a non-team project.
\bigskip

\section{Downfalls}
These are some downfalls of the project, i give my thoughts on what i learned and how i would fix these issues going forward. These two i feel like are actually a huge benefit for me to have a downfall on before i go into industry. I know of these problems from our study but have never actually experienced them properly, this experience should give me insight before going into industry on how i should plan out my work and how valuable Software Methodologies such as Agile are to developing software.

\subsection{Scope Creep}
Scope creep was hard to manage in this project as i kept wanting to add more features before i had a solid foundation. I originally wanted an employee chat and calendar to manage work days, additionally i wanted to add statistics for purchasing habits etc. But all these features made no sense in the scope of this project as i am solo. If i had two addition persons then these features would be feasible, but i wasted a good amount of time researching how i would develop these features instead of just coding the more basic features of the project.

Scope Creep had a very negative impact on this project for this reason, and is something i have learned a great deal amount now that i have actually experienced it.\newline
\newpage
\textbf{Ways i found to combat scope-creep}:
\begin{itemize}
    \item \textbf{Planning}- Carefully planning out the features/components and sticking to that plan and only branching out when the plan is complete or is needed. This would make it less likely to fall prey to scope-creep
    \item \textbf{Focus on Main Features} - Focusing on the main features that i had described in my deign document is the easiest way to combat scope-creep. I had laid out features but over time i focused less and less on those features and branched off to area's and didn't add anything to the actual project.
    
    After this project i now know a lot more on how to focus on the important features, making sure they're implemented and tested adequately before implementing new features.
    \item \textbf{Utilizing Development Cycles} - I feel like i should have utilized Test driven development more, i believe i should of just implemented the feature and gone back to refactor it to save a lot of time. But i mostly ended up not implementing a lot of stuff due to time constraints.
\end{itemize}

\subsection{Utilize Docs}
What i noticed with Kotlin and Android Development, is that there is a huge amount of documentation that i didn't end up reading, so i would implement for example Firebase authentication and Login/Sign-up features through trial and error. But there was a resource available for it for free which i previously thought cost a fixed amount.
So what i learned is that reading the full documentation is a huge benefit before trying to implement a feature.
Documentation for Android, not only helps with implementing features but also have a large amount of tips on UI Design, Marketing Strategies, Performance Tips and many more topics. 

\bigskip
\newpage
\section{Final Conclusion}
This project was a great learning experience. It has given me real insight on how important actual Software Methodologies are to creating Software in a timely manner. I gained experience with a new language and development platform as well. I am actually interested in Android Development as a career after working with Android Studio and Kotlin. The project as a whole just highlights the benefit of Software Development Methodologies and their usage within Industry. Time management is also something highlighted as it played a large role and shows just how important deadlines are within companies. This project provided great experience and knowledge that i know will help me within industry in the future.\newline

As a Final Word, i would like to extend my thanks and gratitude to all of the lecturers at GMIT who have helped me in various ways throughout the last four years. It is thanks to the staff at GMIT that i can go forward into industry with a positive mindset and the ability to learn and adapt to what the industry requires of me. Finally i would like to thank my personal supervisor Martin Hynes and the external examiners who will be grading this project, i hope you find the documentation and code to be of rigorous academic standards.
\bigskip